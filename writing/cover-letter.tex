\documentclass{article}

\usepackage[dvipsnames]{xcolor}
% Measurements are taken directly from the guide
\usepackage[top=1.7in,left=1in,bottom=1in,right=1in]{geometry}
\usepackage{graphicx}
\usepackage[colorlinks=true,
            pdfborder={0 0 0},
            ]{hyperref}
\usepackage{lipsum}
\usepackage[absolute]{textpos}
\usepackage{tikz}
\usetikzlibrary{calc}

% No paragraph indentation
\parindent0pt
\setlength{\parskip}{0.8\baselineskip}
\raggedright
\pagestyle{empty}
% Define SI official colors
\definecolor{UBC_blue}{RGB}{13,40,86}
\definecolor{SIgray}{HTML}{5e6a71}
% Ensure consistency in the footer
\urlstyle{rm}

\usepackage{fancyhdr}
\pagestyle{fancy}
\fancyfoot{}
\fancyhead{}
\fancyfoot[C]{ \thepage}

\setcounter{page}{1}
\renewcommand{\thepage}{Noonan, Cover Letter--- \arabic{page}}

\renewcommand{\footrulewidth}{0pt}
\renewcommand{\headrulewidth}{0pt}
\fancyfoot{}
\fancyfoot[L]{%
    {\footnotesize\color{SIgray}
Department of Biology, Irving K. Barber Faculty of Science, University of British Columbia, Kelowna, B.C., Canada \\[-0.1\baselineskip]
T: 1(250) 258-1118  $\mid$  \href{mailto:michael.noonan@ubc.ca}{michael.noonan@ubc.ca} $\mid$ \url{https://biology.ok.ubc.ca/about/contact/michael-j-noonan/}
}\color{black}}


\begin{document}

% -------------------------------------------------------
% Add logo, the text under the blue line, and the line itself
\begin{textblock*}{2in}[0.3066,0.39](1.6in,1.05in)
    \includegraphics[width=2in]{Lab_Logo_Trans.png}
\end{textblock*}
\begin{textblock*}{6.375in}(1.5in,1.1in)   % 6.375=8.5 - 1.5 - 0.625
    \hfill \color{SIgray} Stefano Mezzini, PhD Biology Candidate\\
    \hfill Department of Biology, UBC-Okanagan

\end{textblock*}
\begin{tikzpicture}[remember picture,overlay]
    \draw[color=UBC_blue,line width=1pt] (current page.north west)+(1.05in,-1.6in) -- ($(-0.625in,-1.6in)+(current page.north east)$);
\end{tikzpicture}


\color{black}
\vspace{10pt}
\hfill Kelowna, Canada, \today{}

Dear Editors,

We are pleased to submit our manuscript entitled ``\emph{Rising temperatures alter how and where boreal mammals move}'' for consideration as a research article in Global Change Biology.

Climate change is rapidly reshaping ecosystems around the world. A growing body of evidence is demonstrating how widespread warming during the last century has caused many terrestrial mammals to change how and where they move, with cascading effects on fitness and community dynamics. Although research on the effects of temperature on mammalian movement behaviour is growing, most studies have focused on documenting changes in only a single species, and few have disentangled temperature effects from seasonal behaviour cycles. Consequently, it is still uncertain how mammals will adapt their movement behaviour throughout the 21\textsuperscript{st} century, nor what the community-level implications of such changes might be. More work is needed on quantfying how different species respond to temperature and the consequences for ecological communities.

Our study addresses this gap by modelling the effects of hourly, proximal temperature on the movement rates and habitat selection of 434 individuals across six mammal species throughout North-Western Canada, namely: boreal and southern mountain caribou, cougars, elk, grizzly bears, mountain goats, and wolves between 1998 and 2023. We employ the flexibility and interpretability of Generalized Additive Models to show that temperature has nonlinear and complex effects on mammalian movement behaviour, but that responses vary greatly across species. We then leverage the estimates to produce quantitative predictions of how climate change will affect each species' movement behaviour throughout the 21\textsuperscript{st} century. We show that, as temperatures rise, many individuals will begin selecting against their current ranges as they adapt to changing phenology and community dynamics. Finally, we present a case study using data on boreal caribou and wolves to show how rising temperatures (and concomitant changes in habitat selection) are likely to increase encounter rates between the two species due to changes in habitat selection, but that the changes are not uniform in space, and there are regions where interactions may decrease, versus others where they will increase. Consequently, rising temperatures are likely to impact wolves' predation of caribou and the effectiveness of wolf reduction strategies.

Our findings offer new insight into the effects of climate change on the movement behaviour of multiple boreal mammalian species, that suggest conservation efforts should account for future changes in movement behaviour as well as any cascading effects on biological communities. Anticipating changes in mammalian movement behaviour will become crucial for effectively and proactively understanding community-level responses and selecting high-quality habitat for long-term conservation. We believe the manuscript will be of strong interest to researchers working at the intersection of climate adaptation, biodiversity protection, and ecological forecasting. All code and data are openly available via GitHub (\url{https://github.com/QuantitativeEcologyLab/bc-mammals-temperature}) to support transparent and reproducible science.

We thank you for considering our submission and look forward to your response.

\vspace{10pt}
Kind regards,

Stefano Mezzini and Prof. Michael Noonan on behalf of all co-authors


\end{document}